\documentclass[10pt,a4paper,danish]{article}
%% Indlæs ofte brugte pakker
\usepackage{amssymb}
\usepackage[danish]{babel}
\usepackage[utf8]{inputenc}
\usepackage{listings}
\usepackage{graphicx}
\usepackage{fancyhdr}
\usepackage{booktabs}
\usepackage{mathpazo}
\pagestyle{fancy}
\fancyhead{}
\fancyfoot{}
			
\rhead{Naja Mottelson\\vsj465}
\lhead{\today}
\rfoot{\thepage}

% Opsæt indlæsning af filer
\lstset{
  language=SQL,
  extendedchars=\true,
  inputencoding=utf8,
  linewidth=\textwidth, basicstyle=\small,
  numbers=left, numberstyle=\footnotesize,
  tabsize=2, showstringspaces=false,
  breaklines=true, breakatwhitespace=false,
}

%% Titel og forfatter
\title{Second Assignment \\ Databases and datammining \\Spring 2012}
\author{Naja Mottelson}

%% Start dokumentet
\begin{document}

%% Vis titel
\maketitle
\newpage

%% HER STARTER RAPPORTEN

\section{Fish Length}
\subsection{Mean and variance}
I have calculated the mean fish length and variance with a small
script in python (contained in the file datamining1.1.py). These
calculations have given me a mean fish length of 2236.25 centimetres
and a mean variance of 1193578.0 centimetres. 

\subsection{Linear regression}
Again, I have programmed both the affine linear model and the mean
squared variance in a minor python script (contained in the file
datamining1.2.py). In the script, I initially collect the input data
(age, temperature) and response data (fish length) from the database,
and turn them into float matrices - respectively a 3x24 matrix padded
with ones\footnote{In correspondance with the prescription given for
  the data matrix in (3.3) in the lecture notes.} and a 1x24 matrix
saved in the variables \texttt{input_matrix} and
\texttt{response_matrix}. I then compute the optimal parameters of the
algorithm described by the formula (3.5) in the lecture notes, saving
the result in the variable \texttt{wT}.

Deliverables: description of software used; mean-squared error;
parameters of the regression model; brief discussion relating
mean-squared-error to the biased sample variance


\section{Proteins in yeast}

\subsection{1-Nearest Neighbour}

\subsection{Dimensionality reduction}


\end{document}